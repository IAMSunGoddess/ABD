
The purpose of CSBD is to describe (1) how a system operates, (2) what security in terms of authentication
and authorization mean for envisioned scenarios in which the system operates, and (3) verify that the system
ConOps (Concept of Operations) satisfies the property of complete mediation. CSBD, until now, described
primarily electronic systems and protocols used in financial services.  Our goal in describing and analyzing
Patrol Base Operations is to see if CSBD can describe other kinds of systems where security and command,
control, and communications (C3) is crucial.\\

For our purposes, our system (patrol base) can be defined as such:

\begin{displayquote}
  ``A patrol base is a position established when a patrol halts for an extended period of time, by means of securing
  and occupying an appropriate location in order to (1) avoid detection, (2) hide, (3) maintain weapons and equipment,
  (4) eat and rest, (5) plan and issue orders, and (6) conduct operations [Ranger Manual, 5-30].''
\end{displayquote}

The configuration of men and materiel changes throughout the various phases of patrol bases. Security is
constantly maintained throughout all configurations. We use CSBD to describe and verify patrol base configurations,
patrol base C3 for configuration changes/phases, and for verifying the security of C3 for each phase or configuration. \\ 

The patrol base operations are clearly defined in the Ranger Handbook (Ranger Handbook 2011) \cite{rangerhandbook} .  Security is both
explicit in the description of the operations and implicit in the command structure and standard operating procedures
of the military. However, the patrol base operations are not defined with regard to systematics verification of the
principle of complete mediation in mind.  That was the aim of this project.  \\

The principle of complete mediation is stated in the well-cited peer-reviewed paper from Saltzer and Schroeder titled
``The Protection of Information in Computer Systems.''

\begin{displayquote}
  ``Every access to every object must be checked for authority... It implies that a foolproof method of identifying the source of every request must be devised... If a change in authority occurs, such remembered results must be systematically updated.'' (1974) \cite{protectInfo} .
\end{displayquote}

For the patrol base operations, we identify the types of objects being accessed and the principals who need to access
them.  For each of these principals, we determine the proper authentication and authorization required to satisfy the
principle of complete mediation.  Complete mediation is verified using the access-control logic (ACL) described in
\emph{Access Control, Security, and Trust: A Logical Approach} \cite{acst}  and  the Higher Order Logic (HOL) theorem prover.\\

We model the patrol base operations as a hierarchy of \emph{secure} state.  These \emph{secure} state machines are defined with
a highly abstract description at the top-level \emph{secure} state machine and decreasing levels of abstraction down through
the ranks to the lower level \emph{secure} state machines.  Lower level abstractions are modularized and based on the higher
level \emph{secure} state machines states. \\

In addition, we adhere to the principle of least privilege.  This principle states that access should be given to the
fewest principals necessary to complete the mission objectives.  This works well with military operations because the
military is accustom to a chain-of-command structure.  



%---- this points LaTeX to PatrolBaseDoc.tex ----
% Local Variables:
% TeX-master: "../PatrolBaseDoc"
% End: