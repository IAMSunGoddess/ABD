The purpose of CSBD is to describe (1) how a system operates, (2) what security in terms of authentication and authorization mean for envisioned scenarios in which the system operates, and (3) verify that the system ConOps (Concept of Operations) satisfies the property of complete mediation. CSBD, until now, described primarily electronic systems and protocols used in financial services.  The purpose of this project was to show that the CSBD approach could be applied to non-automated systems.\\

The patrol base operations were translated from the Ranger Manual into a hierarchy of secure state machines.  These secure state machines were then verified using an access-control logic (ACL) described in \emph{Access Control, Security, and Trust: A Logical Approach} \cite{acst} and the Higher Order Logic (HOL) theorem prover.  \\

Our initial goal was to complete a slice through the entire hierarchy.  This was done in the design phase where the patrol base operations were translated into a hierarchy of \emph{secure} state machines.   Verification of the design phase structures was completed up to the sub-level.  Non-verified aspects of the project were discussed in the Future Work and Research section and remain a work in progress for a master's thesis for one of the authors.\\

Overall, we consider this project to be a great success.  This success could be replicated with other non-automated systems.  In addition, non-automated systems are likely to be automated to some extent in the future--at least in accountability applications.  The techniques employed in this project are directly applicable to such projects where security and accountability of personnel and materials are critical to mission success.  \\

% ---- this points LaTeX to PatrolBaseDoc.tex ----
% Local Variables:
% TeX-master: "../PatrolBaseDoc"
% End: