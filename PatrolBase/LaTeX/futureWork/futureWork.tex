
There were additional areas to continue with work on this project.

\section*{Sub-sub-level}
\label{sec:sub-sub-level}

The sub-sub-level secure state machines were planned, but not verified in ACL and HOL.  The planning
of the sub-sub-level secure state machines included translating them from the Ranger Manual and extensive
discussion about how to verify them in the ACL and HOL.  These discussions included verification of
supporting platoon, squad, and soldier theories (discussed below).  Several diagrams shown in figures
?, ?, and ?  are sufficiently detailed that these levels are ready to be verified in the future.

\section*{Platoon, Squad, and Soldier Theories}
\label{sec:plat-squad-sold}

Platoon, squad, and soldier theories were also planned, but not verified in ACL and HOL.  The plan
was to use these theories to support the sub-sub-level \emph{secure} state machines.  They were to be developed
along with the sub-sub-level \emph{secure} state machines.   Discussion of these theories was sufficient to define
and verify the theories, although their correctness may require some further consultation with the subject
matter expert.   In particular, ideas regarding these theories were discussed but the details were not fully
worked out in terms of their verification in the ACL and HOL.\\

Platoon, squad, and soldier theories were also planned, but not verified in ACL and HOL.  The plan was
to use these theories to support the sub-sub-level \emph{secure} state machines.  They were to be developed
along with the sub-sub-level \emph{secure} state machines.   Discussion of these theories was sufficient to
define and verify the theories, although their correctness may require some further consultation with the
subject matter expert.   In particular, ideas regarding these theories were discussed but the details were
not fully worked out in terms of their verification in the ACL and HOL.

\section*{Integration}
\label{sec:integration}

Integration of the \emph{secure} state machines at the sub-level was the next course of action planned for
this project.  The secure state machines at the sub-level were modularized with a separate theory for each
\emph{secure} state machine.  Integration of these sub-level states was discussed.  There were two approaches
discussed.  The first approach was to integrate the top-level \emph{secure} state machine with the sub-level
\emph{secure} state machines.  This means that the top-level \emph{secure} state machine would transition to
the first state in the sub-level instead of the next state at the top level.  The sub-level \emph{secure}
state machine would step through each state.  When the sub-level reached the COMPLETE state, it would transition
to the next state in the top-level.    With this approach, the input for the top-level would require a list of
input commands (that included all the input commands for the sub-level) before transitioning to the next state
at the top-level.\\

The second approach was to integrate the \emph{secure} state machines at the sub-level and transition from
state machine to state machine and ignore the top-level.  With this approach, transition to the COMPLETE state
in each sub-level \emph{secure} state machine would be required to transition to the first state in the next
\emph{secure} state machine.  \\

The former most method implied a utility to the top-level \emph{secure} state machine, other than a means to
work with decreasing levels of abstraction in the design phase of the project.  But, it made it easier to track
major phases in the overall state of the patrol base operations.  In addition, it only required one integrating
\emph{secure} state machine, rather than one for each level of abstraction. \\

The escape level commands require integration of the \emph{secure} state machine and should be verified once the
integration method is verified.

\section*{Notes to Automation and Authentication}
\label{sec:notes-autom-auth}

It did not escape our attention that the future is automation.  We entertained ideas about automation of the
patrol base operations and how this project could be employed in that area.  In particular, we envisioned an
accountability app worn by soldiers wherein they would scan equipment and indicate the level of readiness.
For example, a soldier would scan his water bottle and indicate that it was at least two-thirds full.  If all
the required equipment was deemed battle ready, then the app would transition to the BATTLE_READY state and
submit the message "SoldierGIJane says battleReady" to her squad.  The squad, likewise, would have an app that
transitioned to the BATTLE_READY state when the squad was ready.  The squad app would send the message "Squad
says squadReady" to the platoon app.  A similar app would exist for the platoon.    A patrol base app would
similarly keep track of the patrol base state based on information sent from the soldiers, leaders, or GPS
devices traveling with the platoon.  All these apps could send information to central command and integrate
that with additional information about the state of all operations.  In addition, analysts could use this
intelligence to track operations and their success and to inform the leaders of the state of all things.
But, of course...this is science fiction (or science future) and outside the scope of this project.

\section*{Adding Complexity}
\label{sec:adding-complexity}

Several additions to the \emph{secure} state machines are possible and may be implemented as this project continues.
For example, there already exists parameterizable \emph{secure} state machines that account for more complex authentication
methods.  Cryptographic methods of authentication involving ID cards, passwords, etc., would be necessary for any
application that automated accountability of personnel and materials.  The current ssm2 parameterizable \emph{secure}
state machine developed prior to this project is similar to ssm1 but accounts for more complicated authentication
techniques.  ssm2 could be overhauled and then applied to some of the sub-sub-level \emph{secure} state machines.
This would serve to work-out the details at the sub-sub-level and prove they work (which they should).\\

Another level of complexity yet involves adding people in roles as principals.  UAV does this.  For example,
this project has the Platoon Leader issuing commands to make transitions among states.  Platoon Leader is a
role and not a person.  Thus, we could increase the complexity by allowing for a soldier with a name, rank,
and other individual information to take-on the role of the Platoon Leader.  In this case, it would be the
soldier representing the Platoon Leader who issues commands.  Such models have also already been described
in a parameterizable \emph{secure} state machine.\\

In addition, we are currently working on implementing the ACL theories in Isabelle/HOL which is much easier
to work with.  Some future theories in this project may be verified in Isabelle/HOL instead of HOL.

% ---- this points LaTeX to PatrolBaseDoc.tex ----
% Local Variables:
% TeX-master: "../PatrolBaseDoc"
% End: