Platoon patrol base (PB) operations are an essential part of many military operations.
They provide Soldiers with a means to avoid enemy contact, rest, resupply, and conduct Squad
level missions (Ranger Handbook 2011, 132). Although some might believe that a PB begins from
an objective rally point (ORP), a PB must be planned for. Our PB software includes a state
machine model for tasks that include planning, movement, as well as conducting both an ORP and PB.\\

Any automation of the patrol base operations (for example Jesse? Professor Chin?) would require
that the design of such system be certified as secure from the start. This approach should be
applied to all systems. Yet, our aim in this project was to show that the certified security by
design approach could be applied specifically to the patrol base operations because it is described
in a manner that is similar to many other military operations. These operations will, no doubt,
be automated in the future. Thus, this project should be viewed as a primer on how to tackle such
an automation in a secure manner, from the start.\\

The certified security by design approach involves taking the design of the automated system and
using access-control logic (ACL) to prove that any action that takes place in the automated system
is correctly executed and has the appropriate authorizations and authentications of those authorities.
Computer-aided reasoning is used to verify that the system is secure. For this project, we used the
higher order logic (HOL) theorem prover. HOL is a trusted and reliable system. It is commonly said by
the HOL community that “HOL is never wrong.”\\

The access-control logic (ACL) was developed by Professor Shiu-Kai Chin and Professor Susan Older of
Syracuse University’s Department of Engineering and Computer Science. Specifics can be found in their
text book titled Access Control, Security, and Trust. The definitions and theorems from the text were
implemented in HOL by Professor Shiu-Kai Chin and Lockwood Morris. For reference a report of the datatypes
, definitions, and theorems that were implemented in HOL are provided in Appendices ?