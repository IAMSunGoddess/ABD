
\section{Design Recommendations}
\label{sec:design-recomm}

The current level of documented states is limited to the top-level and
first sub-level state machines. Exploration of further sub states to
both ensure that the state machine is able to adequately model patrol
base (PB) operations. A state machine model should include states down
to the Soldier level, where applicable. Further sub states have been
identified but, due to time constraints, were not able to be implemented
(see APPENDIX A).\\\\
Additional assurance could be attained by constructing types in Poly/ML
for entities within PB operations. Constructing a Unified Modeling Language
(UML) class diagram with types for Soldiers, units, roles, missions and
their objectives, as well as equipment and supplies could provide a structure
that functions in Poly/ML could check for additional confirmation that a
state is complete (see APPENDIX B). These types would hold Boolean values
as members of types that, when used with logical operations, would prove
whether a state is complete or not. As the state diagram is being completed
down to the individual Soldier level, referencing the UML class diagram and
noting what Boolean members within types would indicate completion of the
state will allow for more complete data structures when implementing the
types in Poly/ML.\\\\
Construction of a Data Flow Diagram (DFD) for sub state machines, i.e.
those below the top-level state machine, provide design cues and illuminate
possible problems. One problem that can be illuminated by use of a DFD is
the need for iterative state execution: i.e. deficiencies are found in a
later state that require a return to a previously completed state. For
example, in our state machine the sub states of PLAN_PB require that
INITIATE_MOVEMENT occur before SUPERVISE. If in the SUPERVISE state and
the PL notices during rehearsals that additional supplies will be needed
to complete the mission, he or she might have to order that the Platoon go
back to the INITIATE_MOVEMENT state and collect up those supplies (see
APPENDIX C, red line labeled “Deficiencies”).\\\\