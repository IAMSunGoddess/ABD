\section{Overall Conclusions}
\label{sec:overall-conclusions}

Overall, the project was a great idea. The initial challenges were in
translating the patrol base operations into a design that could be
implemented using ACL and then verified in HOL. At first, this seemed
like a daunting task. Jesse was well-versed in the patrol base operations
and the methods of the military. Lori was well-versed in the ACL and HOL.
Yet, merging the two was not obvious at first. Once we decided on a state
machine approach to solving the problem, everything ran smoothly. The
beauty of the project was that it could be designed and built into any
system that does check on equipment and the state of readiness of the
patrol base operations at the platoon, squad, and soldier levels.

\section{Future Work}
\label{sec:future-work}

This section discusses future work and other ideas from the perspective of each phase, the design phase and the implementation phase.

\subsection{Design Phase}
\label{sec:design-phase-1}

PB operations were well defined by a state machine. The well-defined
chain of command, roles, and associated tasks yielded easily identifiable
states and sub states that have discrete beginning and ending points.
Design constraints inherent in state machines are manageable at this level
of granularity (i.e. top-level and sub state level).

\subsection{Implementation Phase}
\label{sec:implementation-phase-1}

This section extrapolates on ideas for future work noted in the
various sections above.

\subsubsection{Integration}
\label{sec:integration-2}

One idea that I had an interest in exploring was integrating the
state machines at either the top level or at each individual level.
That is, I was interested in writing a sub-level secure state machine
ssmSL that would transition from one secure state machine to the next.
More specifically, the top-level states were PLAN_PB, MOVE_TO_ORP,
CONDUCT_ORP, MOVE_TO_PB, CONDUCT_PB, and COMPLETE. Each of these states,
save for the terminal state COMPLETE, was implemented at the sub-level
as a secure state machine. Thus, each state in the top-level secure state
machine was itself a secure state machine at the sub-level. These were
each implemented in separate folders and separate “*.sml” files. Each
of these were sub-folders in the “sub level” folder. The idea would be
to generate a separate folder herein and a .sml file that would transition
from the COMPLETE state of one sub-level secure state machine to the
next sub-level secure state machine. And, so on. This could be done at
the sub-sub-level as well. An alternative approach to the integration
problem was to create an integrating secure state machine above the
top-level state machine, in what was called the “OMNI” level state machine.

\subsubsection{Alternatives and Additions}
\label{sec:altern-addit}

At the sub-sub-level we began to sway from the sub levels approach
and consider defining platoon, squad, and soldier theories. These
theories would define Boolean functions or datatypes that would
indicated the degree of readiness of a platoon, squad, or soldier.
For example, the platoon in a sub-sub-level secure state machine
may require that the orders be read back to the headquarters to verify
receipt and correctness. The platoon theory may contain a Boolean
function ordersReadyReady = false (by default). To transition to the
next state in the sub-sub-level state machine, ordersReadyReady = true, would be a condition required. This
would likely be added to the security context list, but defined in the
platoon theory file. Other such functions would be defined in the
platoon, squad, and soldier theories as required to adequately represent
the patrol base. The ideas were hashed out and would be reasonably
straight forward to implement given sufficient time.

\subsubsection{Cryptographic Authentication}
\label{sec:crypt-auth}

Our approach at assuming that “everybody knows who’s who” works
for this project. However, if the patrol base were implemented in
a [add a reference to what Jesse discussed], then authentication
would require a password, key-card, or even a bar code on a soldiers
iPhone. In this case, we would need to take our secure state machine
to the next level and require cryptographic operations on identity.
A parameterizable secure state machine that includes these features
has already been worked out by Professor Shiu-Kai Chin. In analogy
to the ssm11 that I modified from Professor Chin’s ssm1, there already
exists an ssm2. ssm2 extends ssm1 to include cryptographic checks on
identify. It could be used in the same manner as ssm1 (my modified
version was ssm11) without appropriate changes to the project. That is,
it wouldn’t be too much work to make the transition from the
authenticate-by-visual to authenticate-by-password (or other form).
This approach would lend itself well to the idea of [add a reference
to what Jesse discussed], discussed in section “name of section.”